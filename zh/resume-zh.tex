%%%%%%%%%%%%%%%%%%%%%%%%%%%%%%%%%%%%%%%%%
% Medium Length Professional CV
% LaTeX Template
% Version 2.0 (8/5/13)
%
% This template has been downloaded from:
% http://www.LaTeXTemplates.com
%
% Original author:
% Trey Hunner (http://www.treyhunner.com/)
%
% Important note:
% This template requires the resume.cls file to be in the same directory as the
% .tex file. The resume.cls file provides the resume style used for structuring the
% document.
%
%%%%%%%%%%%%%%%%%%%%%%%%%%%%%%%%%%%%%%%%%
%%%%%%%%%%%%%%%%%%%%%%%%%%%%%%%%%%%%%%%%%%
%




%----------------------------------------------------------------------------------------
%	PACKAGES AND OTHER DOCUMENT CONFIGURATIONS
%----------------------------------------------------------------------------------------

\documentclass{resume} % Use the custom resume.cls style
\renewcommand{\baselinestretch}{0.1em} 
\setlength{\parskip}{0.4em}
%\setlength{\smallskip}{0.3em}

\usepackage[left=0.4 in,top=0.2in,right=0.4 in,bottom=0.20in]{geometry} % Document margins
\newcommand{\tab}[1]{\hspace{.2667\textwidth}\rlap{#1}} 
\newcommand{\itab}[1]{\hspace{0em}\rlap{#1}}

\usepackage[colorlinks,
            linkcolor=blue,
            anchorcolor=blue,
            citecolor=green
            ]{hyperref}

\usepackage{soul}
            
            
%!TEX TS-program = xelatex
%!TEX encoding = UTF-8 Unicode    
\XeTeXlinebreaklocale "zh"
\XeTeXlinebreakskip = 0pt plus 1pt

%加這個就可以設定字體
\usepackage{fontspec}

%使用xeCJK
\usepackage[CJKnumber]{xeCJK}  

%設定中文的字型
\setCJKmainfont{Kaiti TC}
\defaultCJKfontfeatures{AutoFakeBold=7,AutoFakeSlant=.2} %以後不用再設定粗斜
\setCJKmainfont[BoldFont=Kaiti TC Bold]{Kaiti TC}

\XeTeXlinebreaklocale "zh" % 針對中文自動換行      


%%%%%%%%%%%%%%%%%%%%%%%%%%%%%%%%%%%%%%%%%%%%
%!TEX TS-program = xelatex
%!TEX encoding = UTF-8 Unicode       

\name{王韋勝} % Your name
\address{ } % Your address
%\address{123 Pleasant Lane \\ City, State 12345} % Your secondary addess (optional)
\address{+886-919-272-449 \\ josixwang@gmail.com \\ \url{https://josix.tw}}  % Your phone number and email

\begin{document}

%----------------------------------------------------------------------------------------
%	OBJECTIVE
%----------------------------------------------------------------------------------------

%\begin{rSection}{OBJECTIVE}

%{Recently graduated, multidisciplinary Engineer with excellent problem solving abilities and process-thinking skill seeks hands on experience within a company that embraces creativity and innovation.Through my studies, I have gained extensive knowledge of production and manufacturing engineering, product design, among many other components of  Mechanical Engineering.Effective communicator who builds positive, cohesive relationship with all level of staff, eager to put my extensive studies to practical, applied use.}

%\end{rSection}

%----------------------------------------------------------------------------------------
%	EDUCATION SECTION
%----------------------------------------------------------------------------------------

\begin{rSection}{\Large 學歷}

{\bf \large 國立政治大學} \hfill {Sep. 2014 - Feb. 2020}
\\ 
\it{資訊科學系\ 及\ 地政學系土地測量與資訊組\quad \it 雙學位}

\begin{rSection}{\Large 工作經歷} % Focus on writing how
\begin{rSubsection}{\large 資訊科學系\ 計算語言學與資訊處理實驗室 CLIP Lab.}{Oct. 2017 - Recent}
{\it{兼任助理}}{}
\item \rm{\bf {\small KKTIX} 售票推薦演算法研究} {\small (KKTIX Cold Start Recommendation)} {\it  Oct. 2017 - Feb. 2019}

{\quad \small -開發 KKBOX 旗下 KKTIX 票務平台的推薦演算法,用於解決新上架商品的推薦冷啟動問題}} 

{\quad \small -設計一套\ {\it Preference-Content Convolution}\ 圖表示學習演算法,並設計 {\it Quey-Based} 的實驗進行演算法驗證}

{\quad \small -在線下測試中推薦精度提升 2.5 倍,並於 2019 年 8 月部署上線}

\medskip
\item {\bf{\small SMU} 合作計畫 - 知識圖譜輔助推薦演算法研究} {\small (Knowledge Graph for Recommendation)}  {\it  Aug. 2019 - Jun. 2020}

{\quad \small -解決推薦情境中使用者/推薦項目資料稀疏的問題}}

{\quad \small -探討並設計能夠處理推薦項目無法對應外部資料情境之聯合學習演算法}

{\quad \small -於此專案設計實驗釐清不同算法於不同控制變因下將如何影響表現,並探討在不同使用程度的使用者上分佈}

\medskip

\item {\bf 線上推薦系統平台建置} {\small (Online Recommendation Web Application)}  {\it  Jun. 2019 - Jan. 2020}

{\quad \small -在應用層面嘗試將現有推薦演算法部署上線,並從工程角度調整算法以達到更好表現}}

{\quad \small -透過此專案我進而了解業界熟悉之工具/概念如 ElasticSearch, ReactJS, SPA 架構等}

{\quad \small -於此專案除了實驗設計及算法的實現外,更著重探討如何將算法導入業界常見的架站模式以建立服務}


\medskip
\item {\bf 實驗室網路管理員} {\small (Server Administration)}  {\it  Aug. 2019 - Recent}

{\quad \small -幫助實驗室同學開設伺服器帳號,更新及管理伺服器軟硬體進行故障偵測}

{\quad \small -SSH 網路安全性設定,並防止惡意暴力登入攻擊導致伺服器效能降低}

{\quad \small -透過 ChrootDirectory 協助開設合作單位 sandbox 環境}

\medskip
\end{rSubsection}

\begin{rSubsection}{}{}
{\it{課程助教}}{}
\item \rm {\bf 計算機程式設計}  {\it  Sep. 2018 - Jun. 2019, Sep. 2020 - Recent}

{\quad \small -提供學生作業諮詢、設計程式作業題目 (包含指標操作、資料結構操作等)}

{\quad \small -實習課教授 C 語言基礎、Vim 編輯器基本指令、Unix-Like 系統操作}

{\quad \small -開發 Online Judge CLI 工具促使學生更加適應 CLI 環境}

{\quad \small -導入 CD 將 Markdown 格式課程內容自動渲染及部署至 GitHub Page}

\medskip

\end{rSubsection} 


\begin{rSubsection}{\large 八拍子股份有限公司\ Rytass Corp.}{Aug. 2017 - Jun. 2018} 
{\it{前端開發實習生}}{}
\item \rm {\bf {\small Volvo App}後台網站}  {\it  Aug. 2017}

{\quad \small -使用 ReactJS、Redux 等前端工具呈現及串接後端 RESTful API 資料}

{\quad \small -熟悉公司共同開發的運作模式及工具如 Git、Git Flow、Code Review 等}

{\quad \small -負責製作後台資料輸入的表單頁面(含表單驗證功能)、列表呈現頁面、頁面樣式調整}

\medskip
\item {\bf 花蓮官方資訊網後台}  {\it  Sep. 2017 - Dec. 2017}

{\quad \small -使用 ReactJS、Redux、Apollo GraphQL 等前端工具串接 Graph API 資料}

{\quad \small -負責後台資料輸入的表單頁面(含驗證功能)、列表呈現頁面(含搜尋、篩選、排序、分頁等功能)、及頁面樣式調整}

{\quad \small -自動測試前台網站是否有錯誤並回報 Issue}

\medskip
\item {\bf {\small CTC} 車商互聯網前台} {\it  Nov. 2017 - Dec. 2017}


{\quad \small -使用 ReactJS、Redux 等前端工具及串接 Restful API 資料}

{\quad \small -負責前台商品內容呈現頁面、樣式調整(含響應式頁面)、商品圖片跑馬燈及燈箱(LightBox)}

\medskip

\item {\bf 浪漫客 - 客委會智慧觀光旅遊平臺後台} {\it  Jan. 2018 - May 2018}

{\quad \small -使用 ReactJS、Redux、Apollo GraphQL 等前端工具串接 Graph API、 RESTful API 資料}

{\quad \small -熟悉打包工具 Webpack 針對 ReactJS 及 JavaScript Babel 設定}

{\quad \small -負責後台資料輸入的表單頁面(含驗證功能)、列表呈現頁面(含搜尋、篩選、排序、分頁等功能)、及頁面樣式調整}
\end{rSubsection} 

\bigskip
\bigskip
\bigskip
\bigskip


\begin{rSubsection}{\large 地政學系 }{Oct. 2016 - July. 2017} 
{\it{兼任助理}}{}
\item \rm {\bf 行動政大 APP 校園地圖功能建置及其他附加應用之研究} {\it  Nov. 2016 - Mar. 2017}

{\quad \small -使用 Django 開發響應式網站、熟悉版本控制}

{\quad \small -負責 Leaflet 製作店家地圖頁面刻畫及網站部署}


\end{rSubsection} 
\end{rSection}

\begin{rSection}{\Large 社群參與及開源貢獻}

\begin{rSubsection}{\large Commitizen-Tools}{Jun. 2020 - Recent}
{\it{貢獻者}}{}
\item \rm 此為協助開發者撰寫符合慣例寫法的 Commit Message 的工具
\item \rm 協助補強 check  message 檢查是否符合慣例寫法並條列不符合者
\item \rm 協助撰寫單元測試
\smallskip
\end{rSubsection}

\begin{rSubsection}{\large 新版 Python Packages Index (warehouse) 網站翻譯}{Sep. 2019 - Recent}
{\it{貢獻者}}{}
\item \rm 協助翻譯網站內容字串
\smallskip
\end{rSubsection}

\begin{rSubsection}{\large PyCon Taiwan 2019, 2020}{Jan. 2019 - Recent}
{\it{開發組志工}}{}
\item \rm 使用 Django 及 Sass 製作 PyCon Taiwan 2019 官方網站所需頁面及樣式更新
\smallskip
\item 負責導航欄(Navigation Bar)、首頁、贊助、售票、傳送門頁面樣式調整 (含響應式頁面)
\end{rSubsection}


\begin{rSubsection}{\large Python 官方說明文件臺灣繁體中文翻譯計畫}{Jun. 2018 - Oct. 2018}
{\it{貢獻者}}{}
\item \rm 負責將 Python 官方 Tutorial 說明文件中第 9 章 Classes 翻譯至繁體中文
\end{rSubsection}

\begin{rSubsection}{\large D4SG 環境廢水防治便民看板}{Jul. 2017 - Jan. 2018}
{\it{志工}}{}
\item \rm 計畫希望減少新竹縣政府、民眾及廠商之間資訊落差,改善稽查效率及透過公開資訊促使各列管廠商自律
\smallskip
\item 負責利用公司統編,使用 Selenium 自動至股票網站爬取公司財報資訊以呈現於看板中
\end{rSubsection}


\begin{rSubsection}{\large Hacktoberfest 2017: Exercism Python 題庫更新}{Oct 2017}
{\it{貢獻者}}{}
\item \rm 更新線上程式教學平台開源專案 Exercism 其 Python 題庫測試程式
\smallskip
\end{rSubsection}

\end

\begin{rSection}{\Large 其他專案}
\begin{rSubsection}{\large Embedding Mixed Queries for Music Recommendation}{Jun. 2018 - Dec. 2018}
{\it{畢業專題}}{}
\item \rm 為多元推薦系統情境,設計採用 Mixed Query 的表示法學習以進行推薦
\smallskip
\item 研讀相關論文算法如 {\it Graph Query Embedding, Heterogeneous Preference Embedding}
\smallskip
\item 實現算法並將其應用於 KKBOX 百萬聆聽記錄資料及 KKBOX Open API 上
\end{rSubsection}

\begin{rSubsection}{\large iCourse 校園選課系統}{Oct. 2017}
{\it{軟體工程課程專案}}{}
\item \rm 使用 React 串接後端 RoR RESTful APIs,並且依照頁面設計刻畫頁面
\smallskip
\item 頁面功能包含:搜尋課程、排序課程、產生課程列表及分頁、產生選課清單列表
\end{rSubsection}

\end


\begin{rSection}{\Large 技能} 
{\bf 程式語言:} {\it 擅長:} {\rm C/C++, JavaScript, ShellScript} $\|$ {\it 熟練:} \rm Python \\
{\bf 函式庫與環境:} \rm PyTorch, ElasticSearch, ReactJS, Django, Unix-Like 環境 \\
{\bf 專長}: \rm 網路爬蟲、資料處理建立 Pipeline、推薦系統、前端開發
\end{rSection} 

\begin{rSection}{\Large 獲得獎項} \itemsep -3pt  
\rm
{2017 臺大黑客松 - HackNTU}{\quad \it 最佳人氣獎} \hfill 2017.07\\
{第三屆全國大專院校暨高中職 StoryMap 校園競賽}{\quad \it 第一名} \hfill 2017.07 \\
{政大資科系畢業專題競賽}{\quad \it 第二名} \hfill 2018.12 \\
\end{rSection} 
\end{document}
